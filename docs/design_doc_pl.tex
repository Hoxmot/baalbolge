\documentclass{article}
\usepackage[utf8]{inputenc}
\usepackage[polish]{babel}
\usepackage[T1]{fontenc}
\usepackage{listings}
\usepackage{hyperref}
\usepackage{multirow}\usepackage{tikz}
\usepackage[normalem]{ulem}
\usepackage[top=3cm, bottom=4.5cm, left=3cm, right=3cm]{geometry}

\hypersetup{
    colorlinks=true,
    urlcolor=cyan,
    linkcolor=blue,
    pdfpagemode=FullScreen,
}

\title{Baalbolge -- Deklaracja języka}
\author{Kamil Bladoszewski}
\date{Kwiecień 2022}

\def\checkmark{\tikz\fill[scale=0.4](0,.35) -- (.25,0) -- (1,.7) -- (.25,.15) -- cycle;} 

\begin{document}

\maketitle

\tableofcontents

\pagebreak

\section{Gramatyka języka}

\begin{lstlisting}
Program. Exps ::= [Exp] ;

EInt.      Exp ::= Integer ;
EBool.     Exp ::= Bool ;
EFunc.     Exp ::= "(" Var [Exp] ")" ;
EInternal. Exp ::= "(" InternalFunc ")" ;
EList.     Exp ::= List ;
EVar.      Exp ::= Var ;
EUnit.     Exp ::= "()" ;

BTrue.  Bool ::= "True" ;
BFalse. Bool ::= "False" ;

LList. List ::= "<|" [Exp] "|>" ;

IFuncDecl. InternalFunc ::= Type Var "[" ArgsList "]" [Exp] ;
ILambda.   InternalFunc ::= Type "[" ArgsList "]" [Exp] ;
IVarDecl.  InternalFunc ::= Type Var Exp ;
ITransact. InternalFunc ::= "transaction" [Exp] ;
IIf.       InternalFunc ::= "if" Exp Exp Exp ;
IWhile.    InternalFunc ::= "while" Exp Exp ;
IWhen.     InternalFunc ::= "when" Exp Exp ;
IConc.     InternalFunc ::= ";" [SameState] ;

SState. SameState ::= Exp ;

AList. ArgsList ::= [Arg] ;

AArg. Arg ::= "(" Type Var ")" ;

TInt.  Type ::= "int" ;
TBool. Type ::= "bool" ;
TVar.  Type ::= "var" ;
TList. Type ::= "<|" Type "|>" ;
TUnit. Type ::= "$" ;

token Var
  ((letter | '_' | '\'' | [".+-/*%?<>=:^;"])
  (letter | digit | '_' | '\'' | [".+-/*%?<>=:^;"])*) ;
comment "#" ;
separator Exp " ";
separator Arg " " ;
terminator SameState ";" ;
entrypoints Exps ;

\end{lstlisting}

\pagebreak

\section{Opis języka}

Język Baalbolge to język imperatywny z elementami funkcyjnymi i składnią inspirowaną językami z rodziny Lisp.

\subsection{Program}

Program składa się z listy wyrażeń. Wyrażaniem jest:

\begin{itemize}
    \item Wartość logiczna: \texttt{True} lub \texttt{False}; Patrz: \ref{typy} Typy
    \item Liczba; Patrz: \ref{typy} Typy
    \item Lista wartości; Patrz: \ref{typy} Typy
    \item Zmienna
    \item Wywołanie funkcji; Patrz: \ref{func:call} Wywołanie funkcji
    \item Wywołanie funkcji spacjalnej; Patrz: \ref{func:special} Funkcje specjalne
    \item Wyrażenie tego samego stanu; Patrz: \ref{same-state} Wyrażenia tego samego stanu
\end{itemize}

Wynikiem programu w języku Baalbolge jest pierwsze wyrażenie głównego ciała, które kończy się wartością inną niż \textit{unit}. Więcej o tym mechaniźmie można przeczytać w sekcji \ref{func:result}.

\subsection{Typy}\label{typy}

Baalbolge obsługiwać będzie typy: \texttt{int}, \texttt{bool}, \textit{unit} (\texttt{\$}), listy, oraz specjalny typ \texttt{var}, który reprezentuje dowolny typ.

\subsubsection{\texttt{var}}\label{type:var}

Dla zwykłych typów, stosowane będzie statyczne typowanie. Niestety, z uwagi na charakterystykę \texttt{var} statyczne typowanie nie jest możliwe i ewentulane sprawdzanie typów (a tym samym zgłaszanie błędów) odbędzie się w runtime.

\subsubsection{Listy}

Listy można dowolnie zagnieżdżać. Gdy korzystamy z typu danych \texttt{var}, lista może być dowolnie zgnieżdżona. W szczególności, zmienna typu \texttt{var} może być listą.

Listy tworzymy poprzez wstawienie wyrażeń oddzielonych spacjami do pomiędzy nawiasy kwadratowe, na przykład: \texttt{<1 2 3>}.

\subsection{Wbudowane operacje na wartościach}

Język ma wbudowane następujące operacje na wartościach:

\begin{itemize}
    \item Dla \texttt{int} -- dodawanie (\texttt{+}), odejmowanie (\texttt{-}), mnożenie (\texttt{*}), dzielenie (\texttt{/}), modulo (\texttt{\%}), porównywanie (wieksze (\texttt{>}), mniejsze (\texttt{<}), równe (\texttt{=}))
    \item Dla \texttt{bool} -- porównywanie przez równość (\texttt{=})
    \item Dla list -- dodawanie elementu na początku (\texttt{:}) lub na końcu (\texttt{+})
\end{itemize}

Wszystkie wbudowane operacje na wartościach będą wywoływane tak jak każda funkcja w języku. Pozostałe standardowe operacje zostaną zaimplementowane w języku Baalbolge i dołączone w bibliotece standardowej:

\begin{itemize}
    \item Dla \texttt{int} -- większe lub równe (\texttt{>=}), mniejsze lub równe (\texttt{<=})
    \item Dla \texttt{bool} -- \texttt{and}, \texttt{or}, \texttt{not}
\end{itemize}

\subsection{Wywołanie funkcji}\label{func:call}

Funkcje wywołujemy poprzedni wpisanie w nawiasach najpierw nazwy funkcji, a następnie listy wyrażeń, których wartości będą argumentami. Argumenty przekazywane są przez wartość. Na przykład:

\begin{lstlisting}
(+ b (+ 3 4))
(func 7 True a <1 2 3>)
\end{lstlisting}

\subsection{Funkcje specjalne}\label{func:special}

Język Baalbolge ma cztery funckje specjalne, które są elementami języka i będą obsługiwane przez interpreter nieco inaczej niż pozostałe funkcje, gdyż wartości podanych tam wyrażeń będą obliczane gdy zajdzie taka potrzeba zamiast, gdy w momencie wywołania funkcji. Do rozważania na przyszłe rozszerzenia języka Baalbolge jest specjalny konstruktor funkcji, który leniwie będzie obliczał wartości argumentów.

Wbudowane funkcje specjalne to:

\begin{itemize}
    \item Deklaracja zmiennych
    \item Deklaracja funkcji
    \item Deklaracja lambdy
    \item \texttt{if}
    \item \texttt{while}
    \item \texttt{when}
    \item Wyrażenia tego samego stanu; Patrz: \ref{same-state} Wyrażenia tego samego stanu
    \item Transakcja; Patrz: \ref{trans} Transakcje
\end{itemize}

\subsubsection{Deklaracja zmiennych}\label{var:decl}

Zmienne deklarujemy w nawiasach, najpierw podajemy interesujący nas typ, następnie nazwę zmiennej, a na końcu wyrażenie, którego wynik będzie wynikiem zmiennej. Wyrażenie obliczane jest w momencie deklaracji zmiennej.

Nazwa zmiennej może zawierać litery, cyfry oraz znaki: "\texttt{.+-/*\%?<>=:;\^}" (cudzysłów nie jest wśród tych znaków). Nazwa zmiennej nie może się jednak zaczynać od cyfry.

Przykładowe deklaracje zmiennych:

\begin{lstlisting}
(int a 1)
(bool b? True)
(int c (+ a 1))
(<int> li <1 2 3>)
(<var> lv <5 False>)
(var v <5 False>)
\end{lstlisting}

Wynikiem deklaracji zmiennej jest \textit{unit}.

\subsubsection{Deklaracja funkcji}\label{func:decl}

Funckje deklarowane są podobnie jak zmienne. W nawiasach podajemy najpierw typ wynikowy, następnie nazwę funkcji, później listę argumentów w nawiasach kwadratowych, a na końcu następujące po sobie wyrażenia.

Deklaracja każdego argumentu wygląda podobnie jak deklaracja zmiennej: w nawiasach podajemy typ argumentu, a następnie nazwę. Różnicą jest brak wyrażenia, które będzie wartością funkcji. Przy późniejszej pracy nad językiem można rozważyć mechanizm domyślnych wartości argumentów, podobny do tego w Pythonie.

Wynikiem funkcji jest wynik pierwszego wyrażenia, który nie jest typu \textit{unit}. Więcej o tym mechanizmie można przeczytać w sekcji \ref{func:result}.

Przykładowe deklaracje funkcji:

\begin{lstlisting}
(int function-1 []
  1)

(int function-name-2 [(int arg1) (bool arg2)]
  (int x 0)
  (if arg2
    x
    (+ x arg1)))
\end{lstlisting}

Wynikiem deklaracji funkcji jest \textit{unit}.

\subsubsection{Lambda}

Lambdy deklarujemy podobnie jak funkcje, ale z różnicą, że nie podajemy ich nazwy:

\begin{lstlisting}
(int [(int x) (int y)]
  (if (< x y)
    y
    x))
\end{lstlisting}

Wynikiem deklaracji lambdy jest lambda.

\subsection{Wynik funkcji}\label{func:result}

Wynikiem funkcji jest wynik pierwszego wyrażenia, które nie jest typu \textit{unit}.

Mechanizm działa następująco: wynik każdego wyrażenia przekazywany jest do miejsca gdzie wyrażenie zostało wywołane. Takie miejsce może spodzieać się wartości (np. deklaracja zmiennej spodziewa się, że wartość wyrażenia będzie wartością zmiennej lub argument w wywołaniu funkcji jest wartością wyrażenia). Jednak w przypadku funkcji, zwracana jest pierwsza wartość wyrażenia, która nie jest \textit{unit}.

Na przykład:

\begin{lstlisting}
(int funkcja []
  (int x 1)
  (int y (+ x 2)
  (+ x y)
  (+ y 5))
\end{lstlisting}

Wynikiem wywołania powyższej funkcji jest 4. Przenalizujemy linijki tego programu:

\begin{enumerate}
    \item Najpierw deklarujemy zmienną \texttt{x} równą 1. Wynikiem deklaracji zmiennej jest \textit{unit} (patrz: \ref{var:decl} Deklaracja zmiennych). Wykonujemy następną linijkę.
    \item Następnie deklarujemy zmienną \texttt{y} równą \texttt{(+ x 2)}. Wynikiem wyrażenia \texttt{(+ x 2)} jest 3. Ta wartość przekazywana jest do deklaracji zmiennej i ponownie deklaracja zmiennej zwraca \textit{unit}.
    \item Wynikiem wywołania jest funkcji jest 4. Jako, że nie jest to \textit{unit}, to wartość jest zwracana przez funkcję.
    \item Poprzednia linijka zwróciła wartość, więc wyrażenie nie jest obliczane.
\end{enumerate}

Podobnie sprawa wygląda z całym programem. Jeżeli któreś z wyrażen w głównym ciele programu zakończy się wartością inną niż unit, to jest to wartość zwracana przez program. Dla przykładu:

\begin{lstlisting}
(int funkcja []
  (int x 1)
  (int y (+ x 2)
  (+ x y)
  (+ y 5))

(int z 3)  
(+ 1 1)
(funkcja)
\end{lstlisting}

Wartość zwracana przez ten program, to 2. Jak wiemy, wywołanie funkcji \texttt{funkcja} zwraca 4, jednak wcześniej wywołane jest \texttt{(+ 1 1)}, które zwraca 2 i jest to wynik programu. Funkcja \texttt{funkcja} nie zostanie obliczona.

W przypadku wcześniejszych wartości, deklaracja funkcji zwraca \textit{unit} (patrz: \ref{func:decl} Deklaracja funkcji), podobnie jak deklaracja zmiennej.

\subsection{Wyrażenia tego samego stanu (obliczanie w tym samym stanie)}\label{same-state}

Wyrażenia można oddzielić od siebie średnikami, wtedy każde obliczenie zostanie w takim samym stanie początkowym, a ewentualne zmiany stanu zostaną naniesione dopiero na koniec wykonywania się całej serii wyrażeń. Takie wyrażenie nazywamy: \textit{wyrażeniem tego samego stanu}.

Na przykład:

\begin{lstlisting}
(int x 0)
(; (int x (+ x 1)) ; (int x (+ x 1)) ; (int x (+ x 1)) ;)
\end{lstlisting}

Po wykonaniu powyższego kodu, w naszym stanie \texttt{x} wynosi 1. Wynika to z tego, że każde z wywołań \texttt{(int x (+ x 1))} wykonywało się w takim stanie początkowym, gdzie \texttt{x} wynosił 0.

Jeżeli wyrażenia będą zmieniały stan na różne sposóby, to stan końcowy będzie ostatnim ze stanów. Na przykład:

\begin{lstlisting}
(int x 0)
(; (int x (+ x 1)) ; (int x (+ x 2)) ; (int x (+ x 3)) ;)
\end{lstlisting}

Po wykonaniu powyższego kodu, w naszym stanie \texttt{x} wynosi 3. Wynika to z faktu, że ostatnie wyrażenie ustawiło taką wartość dla \texttt{x}.

Jeżeli zmieniamy różne elementy stanu (np. różne zmienne), to stan końcowy będzie brał zmiany poszczególnych elementów z różnych wyrażeń składowych. Na przykład:

\begin{lstlisting}
(int x 0)
(int y 0)
(; (int x (+ x 1)) ; (int x (+ x 2)) ; (int y (+ y 3)) ; (int y (+ y 4)) ;)
\end{lstlisting}

Po wykonaniu powyższego kodu, w naszym stanie \texttt{x} wynosi 2, a \texttt{y} wynosi 4.

\begin{enumerate}
    \item Pierwsze wyrażenie składowe ustawia \texttt{x} na 1.
    \item Drugie wyrażenie składowe ustawia \texttt{x} na 2.
    \item Trzecie wyrażenie składowe ustawia \texttt{y} na 3.
    \item Czwarte wyrażenie składowe ustawia \texttt{y} na 4.
\end{enumerate}

Wyrażenie 4 zmieniło \texttt{y}, jednak nie zmieniło \texttt{x}. Stąd patrzymy na wyrażenie 3, jednak ono nie zmieniło \texttt{x}, a jedynie \texttt{y}. Ignorujemy jednak zmienę, \texttt{y}, gdyż mamy już odpowiednią zmianę stanu dla \texttt{y}. Patrzymy na wyrażenie 2, które zmieniło \texttt{x}, stąd zapamiętujemy zmianę stanu dla \texttt{x}. Następnie patrzymy na wyrażenie 1, tkóre zmieniło \texttt{x}, ale zmianę dla \texttt{x} już mamy, więc ignorujemy tę zmianę.

W taki sposób otrzymujemy stan końcowy, gdzie \texttt{x} wynosi 2, a \texttt{y} wynosi 4.

Jeżeli któreś z wyrażeń składowych wyrażenia tego samego stanu kończy się wynikiem innym \textit{unit} (patrz: \ref{func:result} Wynik funkcji), to wykonywanie wyrażeń się zakończy tak, jakby dalszych wyrażeń nie było. Na przykład:

\begin{lstlisting}
(int x 0)
(int x (+ x 1)) ; (+ x 2) ; (int x (+ x 3))
\end{lstlisting}

Wyrażenie \texttt{(+ x 2)} kończy się wynikiem 2, a w stanie końcowym \texttt{x} wynosi 1. Ostatnia składowa wyrażenie tego samego stanu nie miała możliwości się wykonać, ponieważ wcześniejsze wyrażenie zakończyło się wynikiem innym niż \textit{unit}.

Wynikiem wyrażenia tego samego stanu jest \textit{unit} lub wynik pierwszego wyrażenie, który nie jest \textit{unit}.

Na przykład, poniższe wyrażenie tego samego stanu będzie miało wartość \textit{unit}:

\begin{lstlisting}
(int x 0)
(int x (+ x 1)) ; (int x (+ x 1)) ; (int x (+ x 1))
\end{lstlisting}

Z drugiej strony, wartość poniższego wyrażenia tego samego stanu wynosi 2:

\begin{lstlisting}
(int x 0)
(int x (+ x 1)) ; (+ x 2) ; (int x (+ x 3))
\end{lstlisting}

\subsection{Transakcje}\label{trans}

W języku Baalbolge znajdziemy transakcje, które są w stanie zamknąć w sobie pewną grupę wyrażeń. Jeżeli wewnątrz transakcji wystąpi błąd, nie jest on propagowany na cały program, wszystkie zmiany do stanu nie zostają nanoszone i program kontynuuje pracę. Z koleji wszystkie błędy statycznej analizy kodu poza transakcją nie wpływają na wykonanie. Ponadto, możemy ręcznie wykonać rollback transakcji.

W przypadku zwracanego wyniku, transakcja zachowuje się jak funkcja. Jeżeli któreś z wyrażeń składowych zwróci wartość inną niż \textit{unit}, to transakcja kończy się sukcesem, dalsze wyrażenia nie są obliczane i nanoszone są zmiany do stanu.

Jeżeli żadne z wyrażeń nie zwróci wartości innej niż \textit{unit}, to gdy dojdziemy do końca transakcji, transakcja jest commitowana i zwracany jest \textit{unit}.

Jeżeli wykonamy ręczny rollback lub rollback spowodowany błędem, to transakcja zwróci \textit{unit}.

\subsubsection{Błąd przy braku transakcji}

Poniższy kod się nie wykona, ponieważ mamy niezgodność typów przy dodawaniu (\texttt{bool} i \texttt{int}):

\begin{lstlisting}
(int x 0)
(while True
  (int x (+ x 1)))
(bool y True)
(y (+ x y))
\end{lstlisting}

\subsubsection{Brak błędu przy transakcji przed błędem}

Poniższy kod wykona się i zapętli. Choć powinniśmy dostać błąd statycznych typów z dodawania \texttt{int} i \texttt{bool}, to najpierw wykona się transakcja, która ma nieskończoną pętlę \texttt{while}.

\begin{lstlisting}
(int x 0)
(transaction
  (while True
    (int x (+ x 1))))
(bool y True)
(y (+ x y))
\end{lstlisting}

\subsubsection{Błąd przed transakcją}

Poniższy kod zakończy się błędem. Choć mamy transakcję, to pierwszeństwo ma kod wcześniejszy, który rzuca błędem, więc transakcja się nie wykona.

\begin{lstlisting}
(int x 0)
(bool y True)
(y (+ x y))
(transaction
  (while True
    (int x (+ x 1))))
\end{lstlisting}

\subsubsection{Brak błędu przy transakcji okalającej błąd statyczny}

Wewnątrz transakcji dostaniemy błąd statycznego sprawdzania typów, jednak dzięki transakcji zostanie on ignorowany i transakcja się nie wykona, a w jej miejsce wstawiony zostanie \textit{unit}.

\begin{lstlisting}
(int x 0)
(while True
  (int x (+ x 1)))
(transaction
  (bool y True)
  (bool y (+ x y)))
\end{lstlisting}

\subsubsection{Brak błędu przy transakcji okalającej błąd runtime}

W poniższym kodzie z uwagi na użycie \texttt{var}, nie mamy statycznego sprawdzania typów. W runtime dostaniemy błąd przy próbie dodawania \texttt{int} i \texttt{bool}. Jednak dzięki transakcji, błąd zostanie zatrzymany i program będzie się dalej wykonywał.

\begin{lstlisting}
(var x 0)
(transaction
  (var y True)
  (var y (+ x y)))
(while True
  (x (+ x 1)))
\end{lstlisting}

\subsubsection{Ręczny rollback}

Po wykonaniu poniższego kodu, w stanie \texttt{x} wynosi 0, poniważ zmiana stanu została cofnięta.

\begin{lstlisting}
(int x 0)
(transaction
  (int x (+ x 1))
  (rollback))
\end{lstlisting}

\pagebreak

\section{Przykładowe programy}

\subsection{Logiczny OR}

\begin{lstlisting}
(bool or [(bool arg1) (bool arg2)]
  (if arg1
    True
    arg2))
\end{lstlisting}

\subsection{Większy lub równy}

\begin{lstlisting}
(bool >= [(int x) (int y)]
  # Korzystamy z wczesniej zaimplementowanego or
  (or (= x y) (> x y)))
\end{lstlisting}

\subsection{Ciągu Fibonacciego}

\begin{lstlisting}
(int fib [(int n)]
  (int fib-rec [(int f1) (int f2) (int n)]
    (if (= n 0)
      f1
      (fib-rec f2 (+ f1 f2) (- n 1))))
  (fib-rec 0 1 n))

(fib 42)
\end{lstlisting}

\subsection{Silnia}

\begin{lstlisting}
(int factorial [(int x)]
  (int factorial-rec [(int x) (int acc)]
    (if (> x 0)
      (factorial-rec (- x 1) (* acc x))
      acc))
  (factorial-rec x 1))
  
(factorial 10)
\end{lstlisting}

\pagebreak

\section{Tabelka funkcjonalności}

Jako, że chcę, aby mój język łączyć elementy imperatywne z funkcyjnymi, to zamieszczam dwie tabelki. Na samym końcu dopisuję jeszcze elementy mojego języka, którego nie było w tabelkach.

\begin{itemize}
    \item \checkmark oznacza, że zrealizuję dany element
    \item $\pm$ oznacza, że częściowo zrealizuję dany element
    \item ? oznacza, że zastanawiam się jeszcze nad zrealizowaniem elementu
    \item puste miejsce oznacza, że nie zrealizuję danego elementu
\end{itemize}

\subsection{Tabelka języka imperatywnego}

\begin{center}
\begin{tabular}{ c c l }
  \multicolumn{3}{c}{Na 15 punktów} \\
  $\pm$ & 01 & (trzy typy) \\
  \checkmark & 02 & (literały, arytmetyka, porównania) \\
  \checkmark & 03 & (zmienne, przypisanie) \\
  ? & 04 & (print) \\
  \checkmark & 05 & (while, if) \\
  \checkmark & 06 & (funkcje lub procedury, rekurencja) \\
  $\pm$ & 07 & (\sout{przez zmienną} / przez wartość / \sout{in/out}) \\
   & 08 & (zmienne read-only i pętla for) \\
  \hline
  \multicolumn{3}{c}{Na 20 punktów} \\
  \checkmark & 09 & (przesłanianie i statyczne wiązanie) \\
  \checkmark & 10 & (obsługa błędów wykonania) \\
  \checkmark & 11 & (funkcje zwracające wartość) \\
  \hline
  \multicolumn{3}{c}{Na 30 punktów} \\
  \checkmark & 12 & (4) (statyczne typowanie) \\
  \checkmark & 13 & (2) (funkcje zagnieżdżone ze statycznym wiązaniem) \\
  \checkmark & 14 & (1/\sout{2}) (\sout{rekordy}/listy/\sout{tablice}/\sout{tablice wielowymiarowe}) \\
   & 15 & (2) (krotki z przypisaniem) \\
  ? & 16 & (1) (break, continue) \\
  \checkmark & 17 & (4) (funkcje wyższego rzędu, anonimowe, domknięcia) \\
   & 18 & (3) (generatory) \\
\end{tabular}
\end{center}

\subsection{Tabelka języka funkcyjnego}

\begin{center}
\begin{tabular}{ c c l }
  \multicolumn{3}{c}{Na 20 punktów:} \\
  \checkmark & 01 & (dwa typy) \\
  \checkmark & 02 & (arytmetyka, porównania) \\
  \checkmark & 03 & (if) \\
  \checkmark & 04 & (funkcje wieloargumentowe, rekurencja) \\
  \checkmark & 05 & (funkcje anonimowe i wyższego rzędu, częściowa aplikacja) \\
  \checkmark & 06 & (obsługa błędów wykonania) \\
  \checkmark & 07 & (statyczne wiązanie identyfikatorów) \\
  \multicolumn{3}{c}{Listy:} \\
   & 08 & (z pattern matchingiem) \\
   & 09 & (z empty, head, tail) \\
  \checkmark & 10 & (lukier) \\
  \hline
  \multicolumn{3}{c}{Na 25 punktów} \\ 
  \checkmark & 11 & (listy dowolnego typu, zagnieżdżone i listy funkcji) \\
   & 12 & (proste typy algebraiczne z jednopoziomowym pattern matchingiem) \\
  \checkmark & 13 & (statyczne typowanie) \\
  \hline
  \multicolumn{3}{c}{Na 30 punktów} \\
   & 14 & (ogólne polimorficzne i rekurencyjne typy algebraiczne) \\
   & 15 & (zagnieżdżony pattern matching) \\
  \hline
  \multicolumn{3}{c}{Bonus} \\
   & 16 & (typy polimorficzne z algorytmem rekonstrukcji typów) \\
   & 17 & (sprawdzenie kompletności pattern matchingu) \\
\end{tabular}
\end{center}

\subsection{Moje rozszerzenia}

Ponadto, swój interpreter chciałbym rozszerzyć o niewymienione powyżej elementy:

\begin{itemize}
    \item Typ \texttt{var}, sekcja \ref{type:var}
    \item Wrażenie tego samego stanu, sekcja \ref{same-state}
    \item Zwracanie pierwszej wartości wylioczonej w funkcji, sekcja \ref{func:result}
    \item Transakcje, sekcja \ref{trans} -- tutaj może mi nie starczyć czasu
\end{itemize}

\subsection{Suma punktów}

Mam nadzieję, że wszystkie feature'y pozwolą mi uzyskać 30 punktów za interpreter.

\end{document}
